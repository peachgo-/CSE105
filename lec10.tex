
\subsection{Transition between FST and DFA}
% ---------------------------------------------------------------------- 

We can combine an FST with a DFA, obtain a new DFA as result.

The DFA $M'$ recognize regular language $L(M)'$, but the FST $M$ does not recognize a
language, instead it produces a transition function. In general, we use first machine FST
to compute a input, and then fit the output (a string) into the second machine DFA to get
a final state.

\begin{theorem}[Transmition theorem]
    \label{thm:FST_to_DFA} 
    $
    \forall \text{ \upshape FST} \, F = \lst{Q_F, \Sigma,\Gamma, \delta_F, s_F}, \\
    \forall \text{ \upshape DFA} \, D = \lst{Q_D, \Gamma, \delta_D, s_D, F_D}, \\
    \exists \text{ \upshape DFA} \, M = \lst{Q, \Sigma, \delta, s, F} 
    \text{ s.t. \ }
    $
    \begin{align*}
        \overbrace{L(M)}^A &= f^{-1}(\overbrace{L(D)}^B) \text{ meaning }\\
        L(M) &= \set {w \in \Sigma^* \mid f_M(w) \in L(D)} \\
        f(A) &= B \not \leftrightarrow A = \set{x \mid f(x) \in B} 
    \end{align*}
    A is the inverse image of B $\rightarrow A = f^{-1}(B)$.
\end{theorem}

\begin{proof}[Proof of \autoref{thm:FST_to_DFA}] \ \\
    Let $Q = Q_F \times Q_D$,\\
    \phantom{Let}$s\,=(s_F,s_D) $,\\
    \phantom{Let}$F  = Q_F \times F_D$.\\
    Then the transition function 
    $
    \delta \colon Q \times \Sigma \mapsto Q
    $
    will be
    \[
        \delta((q_F,q_D),a) = \big( q_F', \delta_D^*(q_D,w)\big),
        \text{\upshape where }\delta_F(q_F,a) = (q_F',w) 
    \]
\end{proof}

\subsection{Composing FSTs}
% ------------------------------------------------------------

\[
    \xrightarrow{\Sigma^*} FST(M_1)
    \xrightarrow{\Gamma^*}FST(M_2)
    \xrightarrow{\Delta^*} DFA(M)
\]
Any two functions $f \colon \Sigma^* \rightarrow \Gamma^*$ and $g \colon \Sigma^*
\rightarrow \Delta^*$ can be combined using the standard function
composition operation $(g \circ f): \Sigma^* \rightarrow P(\Delta^*)$ defined as 
$(g \circ f)(w) = g(f(w))$. In order for $f_{M_2} \circ f_{M_1}$ to be FST-computable, we
need to be able to run $M_1$ and $M_2$ at the same time, and process the input string $w$
in a streaming fashion. The following theorem shows how to do that

\begin{theorem}[FSTs are Composable]
    \label{thm:composable_FST}
    \begin{align*}
        & \forall \text{\upshape FST} M_1 = \lst{Q_1,\Sigma,\Gamma,\delta_1,s_1},
                                      M_2 = \lst{Q_2,\Gamma,\Delta,\delta_2,s_2}, \\
        & \exists \text{\upshape FST} M = M_2 \circ M_1
        \text{ s.t.\ } f_M = f_{M_2} \circ f_{M_1}.
    \end{align*}
\end{theorem}

\begin{proof}[Proof of \autoref{thm:composable_FST}]
    $
    \text{Let } M = (Q, \Sigma, \Delta, \delta, s)
    \text{ where } Q = Q1 \times Q2, s = (s1, s2) \in Q \text{ and }
    \delta: Q \times \Sigma_1 \mapsto Q \times \Gamma^*_2
    \text{ is the function defined as }
    \delta((q_1, q_2), a) = ((q_1', q_2')), v) \text{ with }
    (q_1',u) = \delta_1(q_1,a)   \text{ and }
    (q_2',v) = \delta_2^*(q_2,u) \text{ and }
    $
    It can be easily verified by induction that
    $f_M = f_{M_2} \circ f_{M_1}$.
\end{proof}

\subsection{FST-reductions}
% ------------------------------------------------------------

The following shows that the preimage of a regular language under an FST-computable
function is also regular.

\begin{definition}[Reduction of languages] \ \\
    For any FST-computable function $f_M \colon \Sigma^* \mapsto \Gamma^*$ and any regular
    language $B \subseteq \Gamma^*$,
    \[
        \text{\upshape the language }A = f_M^{-1}(B) 
        = \set{ w\in \Sigma^* \mid f_M(w) \in B}
        \text{\upshape is also regular.}
    \]
\end{definition}

    This is  called a reduction from $A$ to $B$. If such reduction exists, we say $A$ is
    FST-reducable to $B$, denoted by
    \[
        A \reducable B.
    \]

\begin{theorem}[]\ For any two Language $A,B \subseteq \Sigma^*$, a reduction from $A$ to
    $B$ is a function $f:\Sigma^* \mapsto \Sigma$ s.t. for all $w \in \Sigma^*$,
    \begin{compactitem}
    \item 
        if $w \in A$ then $f(w) \in B$, and
    \item
        if $w \not \in A$ then $f(w) \not \in B$.
    \end{compactitem}
    
    $A$ is a inverse image of $B$, A reduction $f$ is FST-computable if it is computed by
a Finite State Transducer T.  
\end{theorem}

\begin{corollary}
        If  $A \reducable B $ and $B$ is regular, then $A$ is regular,The contrapositive will
\end{corollary}

\begin{corollary}
        If $A \reducable B$ and $A$ is not regular, then $B$ is not regular.
\end{corollary}




