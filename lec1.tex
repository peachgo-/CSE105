
\section{Deterministic Finite Automaton (DFA)}
% ======================================================

A machine consists different drawn in circles with names. Often a state drawn as
a double circle is an ``acceptive state,'' and a plain circle indicates a
rejective state. A machine receives a string consisted of `1's and `0's as input
and the states change as the machine read through input digits. An arrow is used
to indicate which state is to start with. See \autoref{exa:a_DFA} for detailed
information.

\begin{example}[A DFA]
    \label{exa:a_DFA}

    Let's first look at the DFA below which starts at state $A$.
    \centgraph{pics/mp/dfa-0.pdf}

    If the string ``010110'' is input to the machine, will it result in true or false?
    Will the State be acceptive or rejective?
    \[
        \xrightarrow{010110}
        \fbox{M}
        \xrightarrow{ 1/0 \text{(True/False, Accept / Reject)} }
    \]

    There are two arrows leaving state $A$: one with a label reading `1' which points to
    state $B$ and one reading `0' which goes back to state $A$ itself. That means, if an
    input digit reads `1,' the state changes to B, and if `0' the state stays in $A$

    Now step through the procedure:
    \begin{compactenum}
    \item
        The machine starts off at state A with input `0,' which, as explained above,
        changes the state to $A$ itself.
    \item
        Next, the second digit `1' is read so the state is changed to $B$.
    \item
        The next difit `0' makse the state B to switch to state $C$
    \item
        Then state $C$ reads `1' so no state change occurs.
    \item
        The next digit is `1' again so the state remains still on $C$.
    \item
        Last, the digit `0' switches the state from $C$ to $B$.
    \end{compactenum}
    Thus the input string ``010110'' changes the machine to state $B$, which is an
    acceptive state.

\end{example}

\begin{definition}[DFA]
    \label{def:DFA}

    A DFA is a 5-turple
    \[
        M = (Q, \Sigma,\delta,s,F)
    \]
    where
    \begin{compactdesc}
    \item[$Q$]      is a fnite set,     for states
    \item[$\sigma$] is a finite set,    for input alphabet
    \item[$s$]      $\in Q$,            for start states 
    \item[$F$]      $\subseteq Q$,      for accepting states
    \item[$\delta$]
        $Q \times \Sigma \mapsto Q$,
        a function that specifies the transition between states
    \end{compactdesc}
\end{definition}

\begin{example}[DFA table]
    \label{exa:DFA_table}
    According to definition \autoref{def:DFA},
    the machine in \autoref{exa:a_DFA} can be denoted by
    \[
        M = (Q, \Sigma,\delta,s,F)
    \]
    where
    \begin{compactitem}
    \item $Q = \{ A,B,C \}$
    \item $\Sigma = \{ 0,1 \}$
    \item $s = \{ A \}$
    \item $F = \{ B,C \}$
    \end{compactitem}
    And function $\delta$ can be described by the table below.
    \begin{center}
        \begin{tabular}{Mc Mc Mc}
        \hline
        \delta  & 0 & 1 \\
        \hline
        A       & A & B \\
        B       & C & A \\
        C       & B & C \\
        \hline
        \end{tabular}
    \end{center}

\end{example}

\begin{definition}[$f_M$]
    For and DFA $ M = (Q,\Sigma,\delta,s,F) $,
    let
    \[
        f_M: \Sigma^* \mapsto \{ \text{True}, \text{False} \}
    \]
    where $\Sigma^*$ is a set of string over $\Sigma$.

    \[
        f_M(w)
        = \begin{cases}
            \text{True},  & \delta^*(s,w) \in F \\
            \text{False}, & else
        \end{cases}
    \]
\end{definition}

\begin{definition}[$\delta^*$]
    \[
        \delta^* : Q \times \Sigma^* \mapsto Q
    \]
    which is an inductive function defined as
    \[
        \begin{cases}
            \delta^*(q, \varepsilon) &= q \\
            \delta^*(q, aw)          &= \delta^*\( \delta(q,a), w\)
        \end{cases}
    \]
    where $(q \in Q, a \in \Sigma, w \in \Sigma^*)$
\end{definition}
