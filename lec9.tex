

Let's try to prove Non-regular language using the pumping lemma
\begin{example}[Example 1.75 on book $P_{97}$]
    let
    $
        L = \set{ w \in \set{0,1}^* \mid w^{\text{reverse}} = w }
    $
    which is the same as
    $
        L = \set{ ww \mid w \in \set{0,1}^* }
    $,
    means $w \in L$ could be 00100 or 1001. 0011 $\notin L$. 

    We show that $L$ is nonregular, using the pumping lemma.

    \begin{proof} \

        Assume to the contrary that $L$ is regular. Let $p$ be the pumping length given by
        the pumping lemma. Let $s$ be the string $0^p10^p1$. Because $s \in L$ and $s$ has
        length greater than $p$, the pumping lemma guarantees that $s$ can be split into
        three pieces, $s = xyz$, satisfying the three conditions of the lemma.We show that
        this outcome is impossible.

        Condition 3 is crucial, because without it we could pump $s$ if we let $x$ and $z$
        be the empty string. With condition 3: 
        \[
            \underbrace{0 \cdots 0}_p 1 \underbrace{0 \cdots 0}_p 1
        \]
        $\abs{xy}$ must $\le p$, makes $y$ must consist only of 0s, so $xyyz \not \in L$.
    \end{proof}

        Observe that it is critical to the proof that we chose the propriate string, as
        opposed to, say, the string $0^p0^p$ could not lead to a contradiction because it
        is a member of $L$ and it can be pumped.
\end{example}

\begin{example}[Prove $P \notin$ Reg Using Pumping Lemma]
    let
    \[
        P = \set{ w \in \set{1}^* \mid \text{$\abs w$ is a prime number} } 
        = \set {11,111,11111, \cdots}
    \]
    Claim: $P$ is not regular.

    \begin{proof}\

        Assume for contradiction $P$ is regular. \\
        Therefore, by P.L.,
        there is a ``Pumping Length'' $n \ge 1$ s.t.
        1) 2) 3) in Pumping Lemma is true, which means
        all $w \in L$ of length $\abs w \ge n$ can be pumped.

        Let $w = 1^m$ s.t.\ $m \ge n$ and $m$ is a prime,
        Thus $w = 1^m$ satisfies all 3 conditions of P.L.

        Let
        $w = xyz$ s.t.\ $w$ remains satisfying the P.L. \\
        $m = \abs x + \abs y + \abs z $ is a prime. \\
        By P.L.,
        \begin{compactenum}
        \item 
            $
            \forall i \ge 0, \quad
            x \cdot y^i \cdot z = x \overbrace{yy \cdots y}^i z \in L
            $,
        \item
            $
            \abs y \ge 0 ( y \neq \varepsilon)
            $,
        \item
            $
            \abs x + \abs y \le n
            $.
        \end{compactenum}

        \begin{align*}
            x y^i z
            &= 1^{\abs x} \cdot 1^{i \cdot \abs y} \cdot 1^{\abs z}  \\
            &= 1^{\abs x + i \abs y + \abs z}  \\
        .\end{align*}

        ${\abs x + i \abs y + \abs z}$ is also a prime according to the assumption.
        Let
        $i = m+1$,

        \begin{align*}
            &\phteq \overbrace{\abs x + \abs y + \abs z}^m + (i-1)\abs y  \\
            &=      m + (i-1)\abs y  \\
            &=      m + m\abs y  \\
            &=      m(1+\abs y) \not \subseteq \text{ prime number }
            \end{align*}
        Contradicting
        $xy^iz \in L$.
    \end{proof}

\end{example}

\section{Finite State Transducers (FST)}
% ============================================================


\subsection{Defining FST}
% ------------------------------------------------------------

\begin{definition}[Finite State Transducers (FST)]
    A FST is a $5$-tuple
    \[
        M = \lst{ Q,\Sigma,\Gamma,\delta,s }
    ,\]
    where
    \begin{compactitem}
    \item $Q$ is finite set of states,
    \item $\Sigma, \Gamma$ are finite sets of symbols 
            ($\Sigma$ for input, $\Gamma$ for output).
    \item $s \in Q$ is the start state, and
    \item $\delta \colon Q \times \Sigma \mapsto Q \times \Gamma^*$
        is the transition function.
    \end{compactitem}
\end{definition}

\begin{definition}
    \[ 
    \delta^*(q,w) \in Q \times \Gamma^*
    \]
    is the extended transition function defined as such:
    \[
        \begin{cases}
            \delta^*(q,\varepsilon) = (q,\varepsilon) \\
            \begin{matrix}
                \delta^*(q,aw)  =  (q'',uv) \quad \text{ if } \\
                              & =  \delta(q,a)    = (q',u) & \text{ and } \\
                              & =  \delta^*(q',w) = (q'',v) \\
                              & & &=  (\text{ where $a \in \Sigma, u,v,w \in \Sigma^*$}
            \end{matrix}
        \end{cases}
    \]
\end{definition}

\begin{definition}[$f_M$ of and FST]
\[
    f_M(w) = u \quad\text{\upshape s.t.}\quad \delta^*(s,w) = (q,u).
.\]
\end{definition}
