
\subsection{Languages}
% ------------------------------------------------------------
A subset of $\Sigma^*$ of a DFA that contains all inputs to which the output of the
machine is \true is called the \emph{language} of the machine.


In other word, If $A$ is the set of all strings that machine $M$ accepts, we say that $A$
is the \emph{ language of machine $M$} and write $L(M) = A$. ($M$ \emph{recognizes} $A$)

\begin{definition}[Regularity of Language]
    $ L \subseteq \Sigma^* $
    is regular if
    \[
        \exists {\text{\upshape DFA} M} \mid L(M) = L
    \]
\end{definition}

Which means, a DFA could \emph{recognize} L. In short, given a regular language, there
always exist a DFA could be draw.

Notice that 
\begin{itemize}
    \item
        $\varepsilon$(small epsilon) = \emph{ empty string } 
    \item
        $\Sigma$(big Sigma)          = \emph{ alphabet set } 
    \item
        $\varepsilon^*               = \set{\varepsilon}$  
    \item
        $\Sigma^*$                   = $\set{\varepsilon,1,0,10,101,\cdots} = \set{0,1}^*$,
\end{itemize}

\begin{example}[Which of the follwing languages are regular?]
    Given that
    and
    which of the following languages are regular?
    \begin{compactitem}
%     \item
%         $L_1 = \set{ w \in \set{0,1}^* \mid w \text{ is not a multiple of } 3}$,
    \item
        $L_1 = \set{ w \in \set{0,1}^* \mid w \text{ is a power of } 2}$, and
    \item
        $L_2 = \set{ w \in \set{0,1}^* \mid w \text{ is a power of } 3}$.
    \end{compactitem}
    
    $L_1$ is regular while $L_2$ is not. A binary number that is a power of $2$ consists
    of only one $1$ and all other digits should be $0$s. A DFA that recognizes the
    language would be
    \centgraph{mp/dfa-1.pdf}

\end{example}

\begin{definition}[Operations on Languages] \ \\
    \begin{compactdesc}
    \item[Complement]
        $L^C = \set{w \in \Sigma^* \mid w \notin L}$
    \item[Union]
        $L_1 \cup L_2 = \set{w \in \Sigma^* \mid w \in L_1 \vee w \in L_2}$
    \item[Intersection]
        $L_1 \cap L_2 = \set{w \mid w \in L_1 \wedge \in L_2}$
    \item[Concatenation]
        $L_1 \cdot L_2 = \set{w_1 \cdot w_2 \mid w \in L_1, w_2 \in L_2}$
    \end{compactdesc}
\end{definition}

\begin{theorem}
    \label{thm:R_closed_under_union}
    $\mathbb R$ is closed under complement.
\end{theorem}

\begin{example}[If $L$ is regular, is $L^C$ also regular?]
    Yes.

    \begin{proof}[Proof of \autoref{thm:R^C}]
        Let $L \in \mathbb R,$
        prove $L^C \in \mathbb R:$

        By definition, 
        \[
            \exists \DFA \text{ s.t.\ } L(M) = L.
        \]

        Let $M' = \lst{Q,\Sigma,\delta,s,F^C},$

        then $L(M') = L(M)^C = L^C.$

        $L^C \in \mathbb R$ because $L^C = L(M').$
    \end{proof}
\end{example}

\begin{example}[$\forall L_1, L_2$ \\ 
    $
        L_1 \in \mathbb R \vee L_2 \in \mathbb R
        \implies L_1 \cup L_2 \in \mathbb R
    $]
    \label{exa:R_closed_under_union}
        Yes, $\mathbb R$ is closed under union.
\end{example}

