
\section{Nondeterministic Finite Automaton (NFA)}
% ==================================================

In a \emph{nondeterministic} machine, several choices may exist for the next state at
any point.

\subsection{What is NFA}
% ------------------------------------------------------------

Nondeterminism is a generalization of determinism, so every deterministic finite
automaton(DFA) is automatically a nondeterministic finit automaton(NFA). Notice the Difference
between DFA figures and NFA's: 
\begin{compactenum}
    \item 
        NFA may has more than one exiting arrow for symbols in the alphabet.
    \item
        NFA may only only have arrows labeled with members of the alphabet but also
        $\varepsilon$. $(0,1,\cdots,n)$ arrows may exit from each state with the label
        $\varepsilon$.
\end{compactenum}

\begin{example}[How many states needed?]
    \[
        L = \set{w \in \set{0,1}^* \mid
            \text{ the $5^{th}$ (from the left) of $w$ is $0$ } }
    \]
    We will need 7 states

    \centgraph{mp/nfa-0.pdf}

    What if
    \[
    L = \set{w \in \set{0,1}^* \mid
        \text{ the $5^{th}$  from the \emph right) of $w$ is $0$} }
    \]

    \centgraph{mp/nfa-1.pdf}
    Notice this is a \textbf{DFA}, state A has two exit arrows for 0.
\end{example}

\begin{definition}[NFA]
    \label{def:NFA}
    An NFA is a $5$-tuple
    \[
        N = (Q,\Sigma, \delta,s,F)
    \]
    where
    \begin{compactitem}
    \item $Q$ and $\Sigma$ are finite sets
    \item $s \in Q$
    \item $F \subseteq Q$
    \item $\delta \colon
        Q \times \Sigma_\varepsilon
        \footnote{$\Sigma_\varepsilon = \Sigma \cup \set{\varepsilon}$}
        \mapsto \powerset(Q)$
        \footnote{$\powerset(Q)$ = powerset of $Q$}

        $\delta(A,\varepsilon) = \set{H}$
        $\delta(D,\varepsilon) = \emptyset$

    \end{compactitem}

\end{definition}

As said, since a DFA \emph{is} an NFA, the definition of NFA is simply a generalized
version of DFA's. The difference between NFA and DFA regard as  transition function
$\delta$ is in NFA, $\delta$ maps to a set of states instead of exactly one state as of a
DFA

\subsection{Configurations of NFAs}
% ------------------------------------------------------------

\begin{definition}[Configurations of NFA]
    \begin{align*}
        \Conf  &= Q \times \Sigma^* \\
        I_M(w) &= (s,w)             \\
        H_M(w) &= Q \times \set\varepsilon \\
        O_M(q,\varepsilon)
              &= \begin{cases}
              \true,  & if \quad q \in F \\
              \false, & \text{otherwise}
              \end{cases}
              \\
              R &= \set{(q,aw) \mapsto (q',w) \mid
                 \forall q \in Q, a \in \Sigma_{\varepsilon}, w \in \Sigma^*}
                 \end{align*}
\end{definition}

After reading the symbol, an NFA splits into multiple copies of itself and follows
\emph{all} the possibilities in parallel. If the next input symbol doesn't appear on any
of the arrows exiting the state occupied by a copy of the machine, that copy of the
machine dies, along with the branch of the computation associated with if.Finally, if 
\emph{any one} of these copies of the machine is in an accept state at the end of the
input, the \upshape{NFA} accepts the input string. That is to say, if at least one of
these processes accepts, then the entire computation accepts.

\begin{definition}[Computation]
    \label{def:computation}
    \[
    C_0, C_1, \dots , C_n
    \]
    Computation is a sequence of Configurations, such that 
    \[
    C_0 = I(w) \quad 
    \forall i, (C_i, C_{i+1} \in \mathbb{R}) \quad
    [C_i \mapsto C_{i+1}], C_n \in H  
    \]
\end{definition}

an NFA is Accepting if $O(C_n)$ = \true , Rejecting if $O(C_n)$ = \false 

\begin{definition}[Language of NFA]
    \[
        L(N) = \set{w \mid \exists \text{accepting computation on input $w$} }
    \]
\end{definition}

\section{NFA \& DFA}
% ============================================================

According to \autoref{thm:NFA_to_DFA}, and NFA can be translated to a DFA. The method is
to fully expand the NFA and draw out every branch of it, which is explained with more
details in \autoref{dem:NFA_to_DFA}.

\begin{theorem}
    \label{thm:NFA_to_DFA}
    \[
        \forall \NFA,\,
        \exists \text{\upshape DFA} \quad M = \lst{ Q', \Sigma', \delta', s', F'} \text{s.t.}
        L(N) = L(M)
    \]
\end{theorem}

\begin{proof}[Proof of \autoref{thm:NFA_to_DFA}]
    \label{dem:NFA_to_DFA}
    Let $\NFA$ be the NFA recognizing some language $A$,
    we construct a DFA $M = \lst{Q',\Sigma,\delta',s',F'}$ recognizing $A$.
    \begin{align*}
        Q' &= \powerset(Q) \\
        F' &= \set{ A \subseteq Q \mid A \cap F \neq \varnothing} \\
        s' &= E(\set{s})  = \set{q \in Q \mid \exists s
                                                \xrightarrow\varepsilon q_1
                                                \xrightarrow\varepsilon q_2
                                                \xrightarrow\varepsilon q_3
                                                \cdots 
                                                \xrightarrow\varepsilon q
        } \\
        \delta'(A, a) &= E( \operatorname*\bigcup_{q \in A} \delta (q,a) )
%         \footnote{$\operatorname*\cup_{q \in A} =  $}
    \end{align*}
\end{proof}
